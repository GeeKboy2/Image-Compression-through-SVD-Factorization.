\section{Transformations de Householder}
Le but de cette partie est de mettre en place les fonctions nécessaires pour l'utilisation des
matrices de Householder, permettant de mettre en forme certaines matrices.

\subsection{Construction d'une matrice de Householder}

Pour la matrice de Householder, on a l'équation~\ref{eq:householder_matrix},
où $H$ représente la matrice de Householder, puis $N$ un vecteur de taille $n$ et de norme euclidienne $1$.
\begin{equation}
	H = I - 2 \times N \times N^T
	\label{eq:householder_matrix}
\end{equation}

On cherche à trouver comment choisir le vecteur $N$ afin que la matrice de Householder
$H_{U,V}$ envoie $U$ sur $V$. On considère $U$ et $V$ deux vecteurs de même norme.
Si $U^T \times N = 0$, on a $U = V$ et donc $H = I$.
Sinon, on a $N$ colinéaire à $U - V$, et sa valeur est donnée par l'équation~\ref{eq:householder_n}.
\begin{equation}
	N=\pm \frac{U-V}{\|U-V\|}
	\label{eq:householder_n}
\end{equation}

On a donc implémenté une fonction \verb|householder| permettant de calculer $N$ à l'aide de l'équation~\ref{eq:householder_n}, en traitant le cas éventuel où $U$ serait égal à $V$
pour éviter une division par $0$.

Cette fonction permet d'effectuer des changements de bases pour donner des matrices plus simples.
Par exemple, on peut choisir un vecteur $V$ ne contenant qu'une seule valeur non-nulle, la norme de $U$ pour que les deux vecteurs aient la même norme, 
et donc on a la matrice de passage pour le vecteur $U$ associé. C'est d'ailleurs ce que nous utilisons dans les parties suivantes.

\subsection{Produits avec une matrice de Householder}

Ensuite, on a cherché à écrire une fonction optimisée \verb|vector_product| qui calcule le produit d'une matrice de Householder par un vecteur.
En utilisant la formule \ref{eq:householder_matrix}, et par produit avec une matrice colonne $X$, on a l'équation~\ref{eq:vector_product}.
On remarque la présence du produit scalaire $N^T \times X$, ainsi le calcul de $H \times X$ se fait en $\mathcal{O}(n)$ au lieu de $\mathcal{O}(n^2)$.

\begin{equation}
	H \times X = X - 2 \times N \times \underbrace{N^T \times X}_{scalaire}
	\label{eq:vector_product}
\end{equation}

Pour le produit entre une matrice de Householder et un ensemble de vecteurs, nous avons décidé de représenter l'ensemble de vecteurs comme une matrice.
Cette matrice contient alors des vecteurs entre une colonne \verb|start| et une colonne \verb|end|, et des vecteurs nuls pour le reste.

On a donc créé deux fonctions, \verb|matrix_product_left| et \verb|matrix_product_right|, qui toutes deux utilisent le même principe avec le produit scalaire.

La complexité de ces deux fonctions est de $\mathcal{O}(n^2)$, puisqu'on effectue un produit scalaire sur $n$ vecteurs, 
ce qui est bien meilleur qu'un produit matriciel classique de complexité $\mathcal{O}(n^3)$.
