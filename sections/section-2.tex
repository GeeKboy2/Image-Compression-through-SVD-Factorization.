\section{Mise sous forme bidiagonale}
Nous voulons transformer la matrice $A$ en une matrice bidiagonale en utilisant les fonctions \verb|compute_Q1| et \verb|compute_Q2|.
En fait, les matrices $Q_1$ et $Q_2$ annulent les éléments en dessous de la diagonale et au-dessus de la première diagonale supérieure de la matrice $A$, par itérations.

Ces fonctions font appel à la fonction \verb|householder|, qui est appelée sur des sous-matrices de la matrice de départ, pour obtenir la matrice bidiagonale.

En fait, on résout par étapes l'équation \ref{eq:bidiagonalize}, avec $BD$ une matrice bidiagonale, et $Q_{left}$ et $Q_{right}$ des matrices obtenues,
avec respectivement des produits des résultats de \verb|compute_Q1| et \verb|compute_Q2|.

\begin{equation}
	Q_{left} \times BD \times Q_{right} = A
	\label{eq:bidiagonalize}
\end{equation}

La fonction \verb|bidiagonalize| renvoie alors ces matrices $Q_{left}$, $BD$ et $Q_{right}$.
Pour montrer que cette fonction renvoie mathématiquement le bon résultat, il est nécessaire de prouver que les matrices $Q_1$ et $Q_2$ sont bien orthogonales et symétriques.

\bigbreak
\noindent\textbf{Démonstration :}
Prouvons d'abord que ces matrices sont orthogonales.
	Elles sont construites à partir de l'identité, et un bloc est modifié sur la diagonale pour y placer une matrice de Householder.
	Il suffit donc de montrer que les matrices de Householder sont bien orthogonales :
	\begin{align*}
		H &= I - 2 \times N \times N^T\\
		\Leftrightarrow H \times H^T &= (I - 2 \times N \times N^T) \times (I - 2 \times N \times N^T)\\
		\Leftrightarrow H \times H^T &= I - 4 \times N \times N^T + 4 \times N \times \underbrace{N^T \times N}_{= 1} \times N^T = I
	\end{align*}
	
	Ainsi, $H$ est bien orthogonale, et donc par produits par blocs dans les matrices, les matrices $Q_1$ et $Q_2$ sont orthogonales.
	
	Comme $N \times N^T$ est symétrique, alors $H = I - 2 \times N \times N^T$ aussi, et comme $H$ est insérée sur la diagonale d'une matrice carrée, 
	alors la matrice globale (par exemple $Q_1$) est nécessairement symétrique.
	Ainsi, $Q_1$ et $Q_2$ sont symétriques.
\qedsymbol
\bigbreak

Maintenant, il reste à montrer que l'équation~\ref{eq:bidiagonalize} reste vérifiée à chaque itération.

\bigbreak
\noindent\textbf{Démonstration :}
Montrons qu'à chaque itération, on conserve l'égalité $Q_{left} \times BD \times Q_{right} = A$, avec $Q_{left}$ et $Q_{right}$ des matrices orthogonales.

	\medbreak\noindent
	\textbf{Initialisation :} Pour commencer, on a $Q_{left} = I$, $Q_{right} = I$ et $BD = A$. 
	Nécessairement, on a bien $Q_{left} \times BD \times Q_{right} = A$, avec $Q_{left}$ et $Q_{right}$ orthogonales.

	\medbreak\noindent
	\textbf{Hérédité :} Soit $k$ un entier dans $[|0, n - 1|]$. Supposons que l'invariant soit vérifié à la fin de l'itération $k$.
	Montrons alors que l'égalité reste vérifiée pour l'itération $k + 1$.

	L'algorithme consiste à multiplier $Q_{left}$ à droite par $Q_1$, puis $Q_{right}$ à gauche par $Q_2$, et enfin multiplier $BD$ à gauche par $Q_1$ et à droite par $Q_2$.

	Les matrices $Q_{left}$ et $Q_{right}$ sont des matrices orthogonales par hypothèse, et les matrices $Q_1$ et $Q_2$ le sont aussi d'après la démonstration qui précède.
	En utilisant l'hypothèse de récurrence, on a le résultat attendu :
	\begin{equation*}
		Q_{left} \times Q_1 \times Q_1 \times BD \times Q_2 \times Q_2 \times Q_{right} = Q_{left} \times BD \times Q_{right} = A
	\end{equation*}

	En effet, $Q_1$ et $Q_2$ sont orthogonales et symétriques d'après la preuve précédente, donc elles sont auto-inverses : $Q_1 = Q_1^T = Q_1^{-1}$ et $Q_2 = Q_2^T = Q_2^{-1}$,
	ce qui donne $Q_1 \times Q_1 = I$ et $Q_2 \times Q_2 = I$.

	Par produit de matrices orthogonales, $Q_{left}$ et $Q_{right}$ restent orthogonales : l'hérédité est prouvée.

	\medbreak\noindent
	\textbf{Conclusion :} $Q_{left} \times BD \times Q_{right} = A$, l'invariant est donc vérifié.
\qedsymbol
\bigbreak